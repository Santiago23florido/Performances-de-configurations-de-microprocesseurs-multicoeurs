\documentclass[11pt,a4paper]{article}

% ----------------------------------------------------
% PACKAGES ESSENTIELS
% ----------------------------------------------------
\usepackage[T1]{fontenc}
\usepackage[french]{babel}
\usepackage{geometry}
\usepackage{graphicx}
\usepackage{amsmath,amssymb,amsfonts}
\usepackage{siunitx}
\usepackage{caption}
\usepackage{subfig}
\usepackage{booktabs}
\usepackage{float}
\usepackage{xcolor}
\usepackage{fancyhdr}
\usepackage{enumitem}
\usepackage{physics}
\usepackage{titlesec}
\usepackage[strict]{changepage}
\usepackage{framed}
\usepackage{multirow}
\usepackage{algorithmic}
\usepackage[section]{placeins}
\usepackage{tikz}
\usepackage{pgfplots}
\usepackage{textcomp}
\usepackage{cite}
\usepackage{hyperref}

% ----------------------------------------------------
% PARAMETRES DE MISE EN PAGE
% ----------------------------------------------------
\geometry{margin=2.5cm}
\setlength{\parskip}{0.5em}
\setlength{\parindent}{0pt}
\pgfplotsset{compat=1.18}

% ----------------------------------------------------
% COULEURS ENSTA
% ----------------------------------------------------
\definecolor{enstaBleuFonce}{HTML}{003366}
\definecolor{enstaBleuClair}{HTML}{0073CF}
\definecolor{formalshade}{rgb}{0.95,0.95,1}

% ----------------------------------------------------
% CONFIGURATION DES LIENS
% ----------------------------------------------------
\hypersetup{
    colorlinks=true,
    linkcolor=enstaBleuFonce,
    urlcolor=blue,
    citecolor=gray
}

% ----------------------------------------------------
% STYLE DES SECTIONS
% ----------------------------------------------------
\titleformat{\section}[block]
  {\normalfont\Large\bfseries\color{enstaBleuFonce}}
  {\thesection}{1em}{}
  [\vspace{0.3em}\color{enstaBleuFonce}\titlerule\vspace{0.3em}]

\titleformat{\subsection}
  {\normalfont\large\bfseries\color{enstaBleuClair}}
  {\thesubsection}{1em}{}

\titleformat{\subsubsection}
  {\normalfont\normalsize\bfseries\color{black!70}}
  {\thesubsubsection}{1em}{}

% ----------------------------------------------------
% EN-TETES ET PIEDS DE PAGE
% ----------------------------------------------------
\pagestyle{fancy}
\fancyhf{}
\fancyhead[L]{ENSTA Paris}
\fancyhead[R]{TP : Hierarchie memoire et caches}
\fancyfoot[C]{\thepage}
\setlength{\headheight}{16pt}

% ----------------------------------------------------
% ENVIRONNEMENT FORMAL (ENCADRE)
% ----------------------------------------------------
\newenvironment{formal}{%
\def\FrameCommand{%
  \hspace{1pt}%
  {\color{enstaBleuFonce}\vrule width 2pt}%
  {\color{formalshade}\vrule width 4pt}%
  \colorbox{formalshade}%
}%
\MakeFramed{\advance\hsize-\width\FrameRestore}%
\noindent\hspace{-4.55pt}%
\begin{adjustwidth}{}{7pt}%
\vspace{4pt}%
}{%
\vspace{4pt}\end{adjustwidth}\endMakeFramed%
}

% ----------------------------------------------------
% DEBUT DU DOCUMENT
% ----------------------------------------------------
\begin{document}

% ====================================================
% PAGE DE COUVERTURE
% ====================================================
\begin{titlepage}
    \centering
    \vspace*{4cm}

    \includegraphics[width=0.6\textwidth]{figures/ENSTA_LOGP.png}\par\vspace{0.5cm}

    \vspace{0.5cm}
    {\Large Analyse de performances de configurations de microprocesseurs multicoeurs pour des applications parallèles \par}
    \vspace{0.2cm}
    {\huge\bfseries Rapport de Travaux Pratiques \par}
    \vspace{3cm}
    {\Large Luiz Gariglio Dos Santos \par}
    {\Large Helena Guachalla De Andrade \par}
    {\Large Santiago Florido Gomez \par}
    {\Large Franck Ulrich Kenfack Noumedem \par}
    \vspace{1.5cm}
    \textbf{Programme :} Ing\'enieur STIC \par
    \vfill
    Ecole Nationale des Techniques Avancees\\
    ENSTA Paris\\
    Fevrier 2026 \par
\end{titlepage}

% ====================================================
% TABLE DES MATIERES
% ====================================================
\newpage
\tableofcontents
\newpage

% ====================================================
% RESUME
% ====================================================
\section{Resume}

Ce rapport présente une analyse expérimentale des performances d’une architecture multicœurs exécutant un produit matriciel parallélisé avec OpenMP, en mettant l’accent sur l’impact de la hiérarchie mémoire, de la cohérence de cache et des paramètres architecturaux. L’étude utilise deux types de processeurs superscalaires in-order, modélisant des configurations de type Cortex-A7 et Cortex-A15, simulées sous Gem5. Nous analysons l’évolution du nombre de cycles, du speedup, de l’efficacité parallèle et de l’IPC par rapport le nombre de threads et la taille du problème. Une étude complémentaire compare les performances avec et sans cache L2 afin de quantifier l’impact d’un niveau intermédiaire de mémoire sur la scalabilité. Les résultats montrent que, bien que la parallélisation améliore significativement les performances pour des tailles de matrices importantes, le gain reste limité par la fraction séquentielle du code, la contention mémoire et le trafic de cohérence. De cette façon, l’ensemble des résultats met en évidence le compromis fondamental entre parallélisme, hiérarchie mémoire et coûts de synchronisation dans les architectures multicœurs.

% ====================================================
% INTRODUCTION
% ====================================================
\section{Introduction}

L’amélioration des performances des systèmes informatiques modernes repose largement sur le parallélisme multicœur, mais l’augmentation du nombre de cœurs ne garantit pas une accélération proportionnelle des applications. Dans ce travail, nous étudions le comportement d’une application de multiplication de matrices parallélisée avec OpenMP sur une architecture multicœurs simulée. L’objectif est d’analyser les facteurs limitant la scalabilité et de quantifier l’impact des choix architecturaux et de la hiérarchie mémoire.

L’étude se structure autour de trois axes principaux. Premièrement, une analyse du mécanisme de cohérence de cache dans une architecure à bus partagé, afin de comprendre le trafic induit par les lectures et écritures concurrentes. Ensuite, une évaluation expérimentale des performances sur un processeur de type Cortex-A7 (in-order), en étudiant l’évolution des cycles d’exécution, du speedup, de l’efficacité et de l’IPC selon le nombre de threads et la taille des matrices. Enfin, ne comparaison avec un processeur de type Cortex-A15, incluant l’analyse de l’impact de la largeur du pipeline (issue width) sur les performances parallèles. Une étude complémentaire examine l’effet de la suppression du cache L2 afin d’isoler le rôle de ce niveau intermédiaire dans la réduction de la contention mémoire et l’amélioration de la scalabilité.

À travers ces expérimentations, réalisées sous Gem5, nous mettons en évidence les limites du parallélisme OpenMP : poids des sections séquentielles, surcoût des synchronisations, trafic de cohérence et saturation de la hiérarchie mémoire. Les résultats illustrent concrètement la manière dont l'architecture et l’organisation mémoire conditionnent les gains réels obtenus par la parallélisation.

\section{Analyse théorique de cohérence de cache}
\section{Paramètres de l'architecture multicœurs}


\section{Architecture multicœurs avec des processeurs superscalaires in-order (Cortex A7)}

\subsection{Chemin critique, synchronisation et cycles dominants en exécution OpenMP}

En analysant l’architecture multicœur proposée et la manière dont l’algorithme est développé afin de garantir la cohérence des caches, on observe que le processus \emph{master} réalise un travail séquentiel indispensable à l’obtention du résultat final, non parallélisable et donc obligatoire. Ce travail apparaît en une étapes principale , l’initialisation de $A$ et $B$ au moyen de boucles complètes, impliquant davantage d’opérations et, par conséquent, un plus grand nombre d’accès mémoire . Ce traitement n’est pas réparti : il repose sur le cœur exécutant le thread principal. Ainsi, même si les autres cœurs exécutent des calculs en parallèle, ce cœur supporte une « file » de travail supplémentaire. Il convient aussi de noter qu’OpenMP, à la fin d’une région parallèle, impose des barrières de synchronisation : lorsqu’un \emph{worker} termine plus tôt, il reste en attente. 

Dans un programme parallèle, le temps total n’est pas « la somme » des temps de tous les threads ; il correspond au temps nécessaire pour que l’application puisse se terminer. Or, l’application ne peut se terminer que lorsque tous les threads ont fini leur travail . On peut formaliser cette idée par l’équation~\eqref{eq:tapp} :
\begin{equation}
T_{\mathrm{app}} = T_{\mathrm{init}}^{(\mathrm{master})} + \max_{i}\!\left(T_{\mathrm{calc}}^{(\mathrm{thread}\ i)}\right).
\label{eq:tapp}
\end{equation}
Le terme déterminant est le maximum : même si $15$ threads terminent rapidement, si un seul thread prend plus de temps, tous les autres restent bloqués à la synchronisation (\emph{barrier}/\emph{join}) en attendant le dernier. . Ainsi, le master accumule souvent davantage de cycles, car il exécute $T_{\mathrm{init}}^{(\mathrm{master})}$  en plus du calcul parallèle, et il peut aussi inclure des temps d’attente dus aux synchronisations (\emph{barrier}/\emph{join}) ainsi qu’aux effets mémoire et de cohérence. Cela s’accorde avec l’analyse du bus et de la cohérence : lorsque le nombre de cœurs augmente, on observe davantage de lectures sur le bus (matrices partagées $A$ et $B$) et davantage d’invalidations lors des écritures sur $C$, ce qui induit des \emph{stalls} (temps d’arrêt en attente de bus ou d’exclusivité de ligne). 

\subsection{Cycles per configuration}

La Figure~\ref{fig:cycles_per_configuration_cortex_a7} présente le nombre de cycles d’exécution en fonction du nombre de processus exécutés pour les configurations Cortex-A7 disponibles.

\begin{figure}[H]
    \centering
    \includegraphics[width=0.95\textwidth]{../CortexA7/cycles_execution_cortexA7.png}
    \caption{Cortex-A7 : cycles d’exécution de l’application selon le nombre de processus.}
    \label{fig:cycles_per_configuration_cortex_a7}
\end{figure}

Il est proposé de poursuivre l’analyse avec l’estimation du nombre minimal de cycles d’exécution de l’application pour une taille de matrice $M=16$, et ce pour chacune des configurations proposées. Les résultats correspondants sont présentés dans la Figure~\ref{fig:cycles_per_configuration_cortex_a7}. On observe que, même si l’exécution devient plus rapide, le gain de performance n’est pas proportionnel au nombre de threads utilisés, principalement à cause de l’impact des accès mémoire et de la présence des étapes du traitement qui ne sont pas parallélisables et dont le coût devient prépondérant lorsque le temps de calcul parallèle diminue. En particulier, les phases d’initialisation, exécutées séquentiellement par le \emph{master}, constituent une fraction croissante du temps total à mesure que l’optimisation accélère la région de calcul en parallèle. De plus, lorsque le nombre de threads augmente, la quantité d’accès concurrents aux caches partagés (L2/L3) et à la RAM croît également, ce qui génère de la contention et un plus grand nombre de défauts de cache, réduisant ainsi le gain marginal apporté par l’ajout de threads. Par ailleurs, l’écriture concurrente dans la matrice $C$ induit un trafic supplémentaire de cohérence de cache (invalidations et mises à jour de lignes), introduisant des surcoûts absents de la version séquentielle. Enfin, il existe un surcoût propre à OpenMP (création et gestion des threads, ainsi que barrières implicites à la fin du \texttt{parallel for}) qui devient relativement plus significatif lorsque la charge de travail par thread n’est pas suffisamment élevée ou lorsque le \emph{blocking} pour le cache n’est pas mis en œuvre de manière efficace. Par conséquent, le \emph{speedup} observé reflète la combinaison de l’amélioration due au parallélisme et des limites pratiques imposées par la hiérarchie mémoire, ainsi que des coûts de coordination.

Il convient de souligner que, dans le cas où $m$ correspond au nombre de threads, on observe une augmentation du nombre de cycles en passant à $16$ threads par rapport à l’exécution avec $8$ threads. Ce comportement peut s’expliquer par le fait qu’avec $16$ threads, la charge de travail utile par thread devient si faible que le surcoût OpenMP domine l’exécution de la phase parallèle. Cette dégradation est notamment amplifiée par l’augmentation du trafic de cohérence (invalidations et transferts de lignes lors des écritures) ainsi que par une pression accrue sur la mémoire partagée, en particulier le cache L2.

\section{Architecture multicœurs avec des processeurs superscalaires in-order (Cortex A15)}
\subsection{Chemin critique, synchronisation et cycles dominants en exécution OpenMP}
De la même manière que dans le cas du Cortex-A7, le goulot d’étranglement se situe dans l’exécution des sections non parallélisables du code ; par conséquent, c’est le \emph{master} qui détermine le maximum de cycles pour chaque exécution, selon les configurations considérées. Cela reste vrai même si ce cœur n’est pas nécessairement celui qui passe le plus de temps dans l’exécution des sections strictement parallèles. On retrouve ainsi la même conclusion que pour l’autre microprocesseur et l’ensemble de ses configurations.

\begin{figure}[H]
    \centering
    \includegraphics[width=0.95\textwidth]{../CortexA15/cycles_execution_m16_cortexA15_widths.png}
    \caption{Cortex-A15 (\(m=16\)) : nombre de cycles d’exécution selon le nombre de threads et la largeur du pipeline.}
    \label{fig:cycles_execution_m16_cortex_a15_widths}
\end{figure}

\subsection{Speed-Up per configuration}

Dans le but de disposer d’une perspective de comparaison valide, il est proposé d’analyser le rendement du système en parallélisation dans des conditions comparables à celles du Cortex-A7. Ainsi, on vérifie le comportement pour $1$ à $16$ threads avec des matrices de taille $m=16$, en considérant ici l’effet du \emph{largeur de voie} (\emph{issue width}). Il apparaît que, comme dans le cas du A7, pour ces configurations la fraction de calcul parallélisable n’est pas suffisamment représentative au regard des pénalités mémoire (défauts et latences), des surcoûts liés à l’augmentation du nombre de threads (overhead OpenMP) et du poids des sections linéaires non parallélisables. Par conséquent, les \emph{speedups} ne dépassent pas $1.3$ et, de manière attendue, on observe des \emph{speedups} légèrement plus élevés pour les configurations à largeur de voie plus importante, celles-ci permettant de traiter davantage d’instructions par cycle et donc d’améliorer le débit d’exécution lorsque l’application n’est pas strictement bornée par la mémoire.

Afin de comparer ces résultats à un cas où la complexité de calcul augmente, il a également été proposé, comme pour l’étude du Cortex-A15, d’évaluer la parallélisation de $1$ à $16$ threads pour des largeurs de voie de $2$, $4$ et $8$ sur le produit matriciel avec $m=128$. Les résultats mettent en évidence une dynamique comparable à celle observée pour $m=128$ sur le A7 : un comportement initial relativement proche de l’idéal lorsque l’on augmente le nombre de threads, puis une saturation à partir de $8$ threads, avec un écart croissant à la courbe idéale. Cette saturation s’explique par l’augmentation des opérations mémoire et des défauts de cache, par l’overhead associé à un plus grand nombre de threads, ainsi que par le coût relatif des sections linéaires non parallélisables qui devient plus visible lorsque la partie parallèle se réduit. Enfin, on peut conclure que la configuration à largeur de voie $2$ présente un nombre de cycles supérieur aux autres pour l’ensemble des nombres de threads, et qu’on observe globalement une relation inverse entre le nombre de threads et le nombre de cycles requis par l’application. Néanmoins, cette configuration (voie $2$), étant la moins performante en absolu, se révèle aussi plus sensible aux gains de la parallélisation, ce qui se traduit par des \emph{speedups} plus marqués lorsqu’on augmente le nombre de threads, par rapport aux configurations à $4$ et $8$ voies qui, elles, présentent des dynamiques plus proches l’une de l’autre.


Les courbes de \emph{speedup} (référence à \(1\) thread) pour Cortex-A15 sont présentées pour \(m=16\) et \(m=128\) dans la Figure~\ref{fig:speedup_cortex_a15_m16_m128}.

\begin{figure}[H]
    \centering
    \subfloat[\(m=16\) : \emph{Speedup}]{
        \includegraphics[width=0.48\textwidth]{../CortexA15/speedup_m16_vs_1thread_cortexA15_widths.png}
        \label{fig:speedup_cortex_a15_m16}
    }
    \hfill
    \subfloat[\(m=128\) (jusqu’à 16 threads) : \emph{Speedup}]{
        \includegraphics[width=0.48\textwidth]{../CortexA15/speedup_m128_vs_1thread_cortexA15_widths_t16max.png}
        \label{fig:speedup_cortex_a15_m128_t16}
    }
    \caption{Comparaison du \emph{speedup} pour Cortex-A15 selon la largeur du pipeline (\(m=16\) et \(m=128\)).}
    \label{fig:speedup_cortex_a15_m16_m128}
\end{figure}

Les courbes d’efficacité globale associées aux mêmes configurations sont présentées dans la Figure~\ref{fig:efficacite_cortex_a15_m16_m128}.

\begin{figure}[H]
    \centering
    \subfloat[\(m=16\) : Efficacité globale]{
        \includegraphics[width=0.48\textwidth]{../CortexA15/efficacite_m16_cortexA15_widths.png}
        \label{fig:efficacite_cortex_a15_m16}
    }
    \hfill
    \subfloat[\(m=128\) (jusqu’à 16 threads) : Efficacité globale]{
        \includegraphics[width=0.48\textwidth]{../CortexA15/efficacite_m128_cortexA15_widths_t16max.png}
        \label{fig:efficacite_cortex_a15_m128_t16}
    }
    \caption{Comparaison de l’efficacité globale pour Cortex-A15 selon la largeur du pipeline (\(m=16\) et \(m=128\)).}
    \label{fig:efficacite_cortex_a15_m16_m128}
\end{figure}

\subsection{valeur maximale de l’IPC pour chaque configuration}

Pour Cortex-A15, l’IPC peut être analysé selon deux indicateurs complémentaires : l’IPC maximal observé sur un cœur (\emph{max system.cpu*.ipc}) et l’IPC global de la configuration (\(\mathrm{sim\_insts}/\max(\mathrm{numCycles})\)). Les résultats pour \(m=16\) et \(m=128\), en distinguant les largeurs de voie, sont présentés dans la Figure~\ref{fig:ipc_cortex_a15_m16_m128}.

\begin{figure}[H]
    \centering
    \subfloat[IPC maximal par configuration]{
        \includegraphics[width=0.48\textwidth]{../CortexA15/ipc_max_m16_m128_cortexA15_widths.png}
        \label{fig:ipc_max_cortex_a15_m16_m128}
    }
    \hfill
    \subfloat[IPC global par configuration]{
        \includegraphics[width=0.48\textwidth]{../CortexA15/ipc_global_m16_m128_cortexA15_widths.png}
        \label{fig:ipc_global_cortex_a15_m16_m128}
    }
    \caption{Comparaison des métriques IPC pour Cortex-A15 (\(m=16\) et \(m=128\)).}
    \label{fig:ipc_cortex_a15_m16_m128}
\end{figure}

En termes d’IPC maximal par thread dans la simulation, on observe pour $m=16$ une dynamique similaire à celle discutée précédemment : l’IPC maximal atteint par chacun des threads ne dépend pas directement du nombre de threads utilisés par l’application. En revanche, pour une dimension de matrice $m=128$, plusieurs effets peuvent conduire à une diminution de l’IPC maximal des threads. D’une part, la contention en L2 et au niveau de la mémoire augmente lorsque davantage de cœurs demandent simultanément des données, ce qui se traduit par plus de \emph{misses}, davantage de files d’attente et une latence effective plus élevée ; ainsi, malgré la présence de plus de cœurs, chacun passe proportionnellement plus de temps à attendre les données. D’autre part, la cohérence de cache et le trafic associé (et, dans certains cas, le \emph{false sharing}) peuvent dégrader l’IPC lorsque plusieurs threads accèdent à des données proches partageant les mêmes lignes de cache, provoquant des invalidations et un effet de \emph{ping-pong} coûteux. Enfin, l’overhead d’OpenMP augmente avec le nombre de threads : barrières, répartition du travail et synchronisations représentent un coût relatif plus important, en particulier lorsque la charge utile par thread diminue ; avec $16$ threads et $m=128$, la quantité de travail assignée à chaque thread peut devenir suffisamment faible pour que cet overhead pèse fortement sur l’exécution. Néanmoins, du point de vue de l’IPC global, la dynamique reste comparable à celle observée sur le A7, avec des augmentations quasi linéaires de l’IPC lorsque le nombre de threads croît. De plus, lorsque la dimension de la matrice atteint $m=128$, on obtient des IPC globaux sensiblement plus élevés, ce qui traduit une meilleure exploitation du parallélisme et une contribution plus marquée de l’optimisation des opérations de calcul lorsque le nombre de threads employés augmente.

\subsection{Discussion et interprétation}

Dans un scénario où la charge de calcul est faible, l’intégration de multiples threads produit bien un \emph{speedup}, mais la dynamique de réponse du système devient rapidement dominée par les accès mémoire et par la synchronisation. Cela implique que l’effet d’agrégation de threads est moins significatif en termes de nombre de cycles nécessaires pour réaliser la multiplication matricielle. C’est pourquoi, lorsque la taille de la matrice augmente et que le problème devient plus \emph{compute bound}, on commence à obtenir de meilleurs \emph{speedups}, comme on l’observe également avec le microprocesseur Cortex-A15.

Par ailleurs, l’augmentation de la largeur des voies (par exemple de $2$ à $4$ puis $8$) peut être reliée à une diminution du nombre de cycles requis pour exécuter le produit matriciel, puisque davantage d’instructions peuvent être traitées par cycle. Toutefois, précisément parce que cette optimisation réduit le temps d’exécution de base, les gains relatifs associés à la parallélisation ont tendance à diminuer pour les configurations à $4$ ou $8$ voies : une fraction plus importante du temps total est alors expliquée par des coûts non idéaux (attentes mémoire, cohérence et synchronisation) plutôt que par le calcul utile.

Enfin, on constate que, pour $8$ threads et une matrice de taille $M=128$, les IPC globaux obtenus ne sont que légèrement supérieurs à ceux mesurés sur les configurations de type A7. Cela s’explique par le fait que, compte tenu de la nature de l’opération (produit matriciel) et de la taille considérée, l’exécution devient rapidement \emph{memory bound} et dépend fortement des surcoûts liés à la synchronisation et à l’exécution OpenMP. En conséquence, les avantages micro-architecturaux du A15, mis en évidence lors du premier TP, se trouvent nettement atténués dans ce contexte, ce qui rend la différence entre l’utilisation d’un A7 et d’un A15 relativement peu significative pour cette application.


\section{Configuration and conclusion}
\subsection{Configuration CMP la plus efficace}
Se propose donc la sélection d’un microprocesseur \textbf{Cortex-A7 (ARM)}, avec une configuration à \textbf{8 threads}, car cette option offre un compromis pertinent entre le nombre de threads, le \emph{speedup} et l’IPC global observé. Ce choix est principalement justifié par le fait que, dans les conditions de l’application étudiée, l’exécution cesse rapidement d’être \emph{compute bound} pour devenir fortement contrainte par la mémoire ; dans ce contexte, l’emploi d’un microprocesseur A15 n’apporte pas d’amélioration significative. En effet, les résultats montrent non seulement que les \emph{speedups} obtenus avec les A15 pour une matrice de taille $M=128$ restent proches de ceux mesurés sur A7, mais aussi que, pour le même nombre de threads ($8$), les valeurs d’IPC global sont très similaires à celles de la configuration proposée. Par ailleurs, la configuration à $8$ threads est retenue de préférence à celle à $16$ threads, car l’on observe que, pour des tailles de matrice inférieures à $M=128$, le \emph{speedup} tend à saturer, notamment en raison des phénomènes déjà identifiés : défauts de cache et pénalités mémoire, surcoûts OpenMP, coûts de synchronisation, ainsi que la présence de sections linéaires non parallélisables du code qui limitent le gain global.


\subsection{speedup supra-linéaire}

Un \emph{speedup} supra-linéaire peut apparaître lorsque l’augmentation du nombre de threads améliore fortement l’utilisation des caches. En effet, les données peuvent être réparties entre plusieurs caches (notamment L1 et, selon l’architecture, L2), ce qui réduit la fréquence des accès à la mémoire principale et diminue la latence effective des lectures/écritures. Dans ce cas, le temps total peut baisser plus vite que la simple division du travail entre threads, car on gagne simultanément en parallélisme et en localité mémoire. Ce phénomène est généralement favorisé par des matrices de grande taille, des caches relativement petits (rendant la localité plus sensible), et un nombre de threads modéré, pour lequel l’augmentation de capacité de cache agrégée ne s’accompagne pas encore d’une contention excessive ni de surcoûts de synchronisation dominants.


\subsection{Conclusion}

Ce TP met en évidence l’impact du parallélisme et du type de cœur sur les performances. Les cœurs \emph{in-order} offrent un bon compromis pour concevoir des CMP efficaces en surface et en consommation, tandis que les cœurs \emph{out-of-order} (o3) permettent d’augmenter l’IPC grâce à une meilleure exploitation du parallélisme au niveau des instructions. Néanmoins, le \emph{speedup} observé reste souvent limité par la hiérarchie mémoire (contention, défauts de cache, latences) et par les surcoûts de synchronisation et de coordination imposés par l’exécution parallèle, ce qui empêche d’atteindre un gain proportionnel au nombre de threads.
\begin{thebibliography}{00}
\bibitem{Profiling} M. J. P. (University of York), ``Profiling,'' \emph{Lecture Notes (4th Year HPC), University of York}. [Online]. Available: https://www-users.york.ac.uk/\textasciitilde{}mijp1/teaching/4th\_year\_HPC/lecture\_notes/Profiling.pdf. Accessed: Feb. 9, 2026.
\end{thebibliography}

\end{document}
